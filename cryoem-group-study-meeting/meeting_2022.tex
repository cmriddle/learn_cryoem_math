\documentclass[11pt, oneside]{article}   	% use "amsart" instead of "article" for AMSLaTeX format
\usepackage{geometry}                		% See geometry.pdf to learn the layout options. There are lots.
\geometry{letterpaper}                   		% ... or a4paper or a5paper or ... 
%\geometry{landscape}                		% Activate for rotated page geometry
%\usepackage[parfill]{parskip}    		% Activate to begin paragraphs with an empty line rather than an indent
\usepackage{graphicx}				% Use pdf, png, jpg, or eps§ with pdflatex; use eps in DVI mode
								% TeX will automatically convert eps --> pdf in pdflatex		
\usepackage{amssymb}
\usepackage{hyperref}
\usepackage{multirow}
\usepackage{ulem}
\usepackage{listings}
\usepackage{url}
\usepackage{longtable}

\lstset{frameround=fttt,language=Python,breaklines=true}


%SetFonts


\title{Online CryoEM Study Group}
\author{Geoffrey Woollard, Vancouver, Canada}
%\date{}							% Activate to display a given date or no date

\begin{document}
\maketitle

\tableofcontents



\pagebreak
\section{Dates and Topics}
Meetings are generally on Thursdays (morning Pacific time, afternoon Eastern time, evening Europe)

\begin{center}
\small
 \begin{longtable}{|| c c p{90mm} ||} 
 \hline
 Date & Time  & Topic \\ [0.5ex] 
 \hline\hline
 Th 20 Jan 2022 & \tiny{8:30 AM PST}  & advanced rotations \\ 
  \hline
 Th 27 Jan 2022 & \tiny{8:30 AM PST}  & Guest: Qinwen (Wendy) Huang \\ 
[1ex]  
 \hline
 Th 3 Feb 2022 & \tiny{8:30 AM PST}  & mathy cryoem paper \\ 
 \hline
 Th 10 Feb 2022 & \tiny{8:30 AM PST}  & advanced rotations \\ 
 \hline
 Th 17 Feb 2022 & \tiny{8:30 AM PST}  & mathy cryoem paper \\ 
 \hline
 Th 24 Feb 2022 & \tiny{8:30 AM PST}  & advanced rotations \\ 
 \hline
 Th 3 Mar 2022 & \tiny{8:30 AM PST}  & mathy cryoem paper \\ 
 \hline
 Th 10 Mar 2022 & \tiny{8:30 AM PST}  & advanced rotations \\ 
 \hline
 Th 17 Mar 2022 & \tiny{8:30 AM PST}  & mathy cryoem paper \\ 
 \hline
 Th 24 Mar 2022 & \tiny{8:30 AM PST}  & advanced rotations \\ 
 \hline
\end{longtable}
\end{center}

\pagebreak
\section{General Information}

\subsection{Archived material}

Meetings from 2020-2021 are archived \href{https://github.com/geoffwoollard/learn_cryoem_math/blob/master/cryoem-group-study-meeting/meeting.pdf}{here}. The audience was a mix of beginners and advanced practitioners, and computational methods developers. 

\subsection{Audience and Streams}

Feel free to share this document and direct people to sign up at \url{https://forms.gle/BUeUW14vV4pyQbDDA} so I have the emails in one place. Online meeting links are emailed to those on this list. {\bf Please join the Slack group and ask questions there, rather than emailing me.}

\subsection{Audience}
In 2022, I am catering to a computational methods development audience. I see this group as a way for computational methods developers to get together in a "pre-competitive" learning environment.

{\bf Practitioners $\rightarrow$ computational methods developers}: You are a structural biologist, or biochemist, and perhaps an advanced cryo-EM practitioner. You would like to train in computational methods development, either to do very advanced data processing, or develop your own methods.

{\bf Pure computational discipline $\rightarrow$ cryo-EM computational methods developers}: You have a background in computer science (computer vision, deep learning, statistics, electrical/computer engineering) and would like to develop methods for the "killer application" of cryo-EM.

\subsection{Pre-requisites}
The bar is quite high, and this group is not for all. There are very good resources out there for self-study; see this \href{https://github.com/geoffwoollard/learn_cryoem_math/blob/master/README.md}{annotated a bibliography}. If you have done an undergraduate degree in an advanced computational program (physics, chemistry, computer science, statistics, applied math) or are a PhD student in a computational field, then you are in good company in this group.

\subsection{Scope}

\subsubsection{Math / Computer science}
\begin{enumerate}
	\item amortized inference, model learning
	\item physics aware/inspired/infused deep learning
	\item deep learning of the image formation model (rotation, etc)
	\item computationally modelling uncertainty in the image formation model
	\item geometric deep learning and invariance/equivariance in cryo-EM
	\item computational optimal transport
	\item computational differential geometry
	\item optimization
	\item custom GPU kernels, including gradient for backprop/autodiff
\end{enumerate}

\subsubsection{Physics}
\begin{enumerate}
	\item Electron optics
	\item Higher order CTF aberrations
	\item Multi-slice
	\item Sample damage
	\item Detector physics
	\item Solvation
	\item Poisson-Boltzmann equation
	\item Modelling choices to encode coulombic density
\end{enumerate}

\subsection{Meeting Format}
The meetings are meant to be more informal that is typical in research talks. The point is to learn and discuss with other learners, experienced practitioners, and experts. They are also more comprehensive than typical journal clubs. We may stick with a papers or series of papers for multiple weeks to sufficiently learn the material.

\subsection{Slack}
We will use the Slack channel 'cryoem\_study\_group' for asynchronous chat. Please join the Slack group and ask questions there, rather than emailing me. You can request a link to join by emailing me.

\subsection{Testimonials}

\begin{itemize}
	\item {\it Shayan Shekarforoush, PhD student with Marcus Brubaker and David Fleet, Jan 2021}. I joined this reading group in mid October and I wish I would have done so much earlier. Although I joined when the group was in the middle of reading a fascinating, recently published book in Single-Particle cryoEM, everyone was so welcome that I did not feel I am way behind others. My background is in CS and I do research as a method developer in this field. With that said, I learned a lot from discussions of people with expertise in experimental side of this area. I believe that this group helped me to build a better intuition and now I feel more comfortable with the underlying math and physics of this topic. This group also provided the opportunity to attend talks of prominent researchers in cryoEM where anyone could openly ask their questions and have clear discussions. Looking forward to having more collaborations with the members of this group.
\end{itemize}

\section{Learning Resources}


\begin{enumerate}
	\item A good place to start is Grant Jensen's \href{https://www.caltech.edu/about/news/grant-jensen-cryo-em}{popular} online course \href{https://jensenlab.caltech.edu/courses/}{'Getting Started in Cryo-EM' }.

	\item After getting up to speed on the prerequisites, another good next step is the content developed by Dr Frederick Sigworth and others at \url{https://cryoemprinciples.yale.edu/video-lectures}. If you are having problems with links, then try viewing his content on YouTube.

	\item  Also, I highly recommend the interactive learning material developed by \href{http://cryoem.tudelft.nl/group/arjen-jakobi/}{Arjen Jakobi}, for a course on \href{https://gitlab.tudelft.nl/aj-lab/teaching/-/wikis/NB4020}{High-Resolution Imaging at TUDelft}: "The practicals are computational assignments in the form of interactive Jupyter notebooks hosted in a virtual learning environment. These notebooks contain code that can be executed to perform certain tasks or visualize results; you do not need any active knowledge of coding to work through the notebook." For the curious, the code that generates the visualizations is available on the repository.
 
	\item I have made an annotated bibliography organized thematically \href{https://github.com/geoffwoollard/learn_cryoem_math#resources}{here}.

	\item Coding notebooks to play around with are \href{https://github.com/geoffwoollard/learn_cryoem_math/tree/master/nb}{here}. If there is incompatibility between the notebook and the \href{https://github.com/geoffwoollard/learn_cryoem_math/tree/master/src}{code base} in the repo, that is because the code base has been updated. Older version of the code are available via past commits.
\end{enumerate}

\pagebreak
\section{Upcoming Meetings}


\subsection{2022 - advanced treatment of rotations}

\textendash {Pre-reading}
\begin{enumerate}
	\item (2020). A Smooth Representation of Belief over SO(3) for Deep Rotation Learning with Uncertainty. \url{https://arxiv.org/pdf/2006.01031.pdf}
	\item (2021). Eliminating Topological Errors in Neural Network Rotation Estimation Using Self-selecting Ensembles. \url{https://dl.acm.org/doi/pdf/10.1145/3450626.3459882}
	\item (2021). On the Continuity of Rotation Representations in Neural Networks. \url{https://arxiv.org/pdf/1812.07035.pdf}
	\item (2013). Rotation Averaging. \url{https://link.springer.com/content/pdf/10.1007/s11263-012-0601-0.pdf}
	\item (2021). Learning Rotation Invariant Features for Cryogenic Electron Microscopy Image Reconstruction. \url{https://arxiv.org/pdf/2101.03549.pdf}
	\item (2020). SE(3)-Transformers: 3D Roto-Translation Equivariant Attention Networks. \url{https://arxiv.org/pdf/2006.10503.pdf}
	\item Falorsi, L., de Haan, P., Davidson, T. R., \& Forr�, P. (2020). Reparameterizing distributions on Lie groups. AISTATS 2019 - 22nd International Conference on Artificial Intelligence and Statistics, 89. \url{https://arxiv.org/pdf/1903.02958.pdf}
\end{enumerate}
%\textendash {Questions}
%\begin{enumerate}
%	\item 
%\end{enumerate}
%\textendash{Meeting Recording} \\
%{\tiny \url{https://ubc.zoom.us/rec/share/fZzzdjCxKEk6GWPKzTKSr4ngCfJoja_OxoJGX7x0_y4rscGfvcUh4YUXxZnNKBZc.rPCBs5661FbNTXC4}} \\
%Access Passcode: \texttt{jT1\%n6Oj}


\subsection{2022 - assorted mathy cryoem papers}

\textendash {Pre-reading}
\begin{enumerate}
	\item Tagare, H. D., Kucukelbir, A., Sigworth, F. J., Wang, H., \& Rao, M. (2015). Directly reconstructing principal components of heterogeneous particles from cryo-EM images. Journal of Structural Biology, 191(2), 245?262. \url{http://doi.org/10.1016/j.jsb.2015.05.007}
	\item Zivanov, J., Nakane, T., \& Scheres, S. H. W. (2019). A Bayesian approach to beam-induced motion correction in cryo-EM single-particle analysis. IUCrJ, 6(1), 5?17. \url{http://doi.org/10.1107/S205225251801463X}
	\item Katsevich, E., Katsevich, A., \& Singer, A. (2015). Covariance matrix estimation for the cryo-em heterogeneity problem. SIAM Journal on Imaging Sciences, 8(1), 126?185. \url{http://doi.org/10.1137/130935434}
	\item Penczek, P. A. (2010). Resolution Measures in Molecular Electron Microscopy. In Methods in Enzymology (1st ed., Vol. 482, pp. 73?100). Elsevier Inc. \url{http://doi.org/10.1016/S0076-6879(10)82003-8}
	\item Ede, J. M. (2020). Review: Deep learning in electron microscopy. ArXiv. \url{http://doi.org/10.1088/2632-2153/abd614}
	\item  Qinwen Huang, Ye Zhou, Hsuan-Fu Liu, \& Alberto Bartesaghi (2021). Weakly Supervised Learning for Joint Image Denoising and Protein Localization in Cryo-EM \url{https://www.mlsb.io/papers_2021/MLSB2021_Weakly_Supervised_Learning_for.pdf}
\end{enumerate}

\subsection{2022 - advanced microscopy}

\textendash {Pre-reading}
\begin{enumerate}
	\item Glaeser, R. M., Hagen, W. J. H., Han, B. G., Henderson, R., McMullan, G., \& Russo, C. J. (2021). Defocus-dependent Thon-ring fading. Ultramicroscopy, 222(October 2020), 113213. \url{http://doi.org/10.1016/j.ultramic.2021.113213}
	\item Russo, C. J., \& Egerton, R. F. (2019). Damage in electron cryomicroscopy: Lessons from biology for materials science. MRS Bulletin, 44(12), 935?941. \url{http://doi.org/10.1557/mrs.2019.284}
	\item Russo, C. J., \& Henderson, R. (2018). Ewald sphere correction using a single side-band image processing algorithm. Ultramicroscopy, 187, 26?33. \url{http://doi.org/10.1016/j.ultramic.2017.11.001}
	\item Electron optics textbook chapters (Hawkes and Kasper; Spence; Reimer and Kohl)
\end{enumerate}

\subsection{27 Jan 2022 - computational methods development - Qinwen (Wendy) Huang}

\textendash {Pre-reading}
\begin{itemize}
	\item  Qinwen Huang, Ye Zhou, Hsuan-Fu Liu, \& Alberto Bartesaghi (2021). Weakly Supervised Learning for Joint Image Denoising and Protein Localization in Cryo-EM \url{https://www.mlsb.io/papers_2021/MLSB2021_Weakly_Supervised_Learning_for.pdf}
\end{itemize}
\textendash {Questions}
\begin{enumerate}
	\item This paper exploits analytical likelihood-prior-posterior conjugacy between gaussian distributions. How could we extend this to Poisson noise with Poisson-Gamma conjugacy? What would be Poisson, and what would be Gamma?
\end{enumerate}

%\subsection{20 May 2021 - Sharpen - Electron Optics: Lens Aberations }
%\textendash{Pre-reading}
%\begin{enumerate}
%	\item "2.3 Lens Abberations" (pp. 31-40) in Reimer, L., \& Kohl, H. (2008). Transmission Electron Microscopy Physics of Image Formation. Springer series in optical sciences (Vol. 51). http://doi.org/10.1007/978-0-387-34758-5
%	\end{enumerate}
%%\textendash {Questions}
%%\begin{enumerate}
%%	\item Install 
%%\end{enumerate}

%\subsection{3 June May 2021 - Sharpen - Electron Optics: Electron Waves and Phase Shifts }
%\textendash{Pre-reading}
%\begin{enumerate}
%	\item "3.1 Electron Waves and Phase Shifts" (pp. 45-55) in Reimer, L., \& Kohl, H. (2008). Transmission Electron Microscopy Physics of Image Formation. Springer series in optical sciences (Vol. 51). http://doi.org/10.1007/978-0-387-34758-5
%	\end{enumerate}
%%\textendash {Questions}
%%\begin{enumerate}
%%	\item Install 
%%\end{enumerate}

\pagebreak
\section{Past Meetings}
\pagebreak




\end{document}  